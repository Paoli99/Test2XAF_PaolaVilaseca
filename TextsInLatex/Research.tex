\documentclass{article}


\usepackage[spanish]{babel}

% Set page size and margins
% Replace `letterpaper' with `a4paper' for UK/EU standard size
\usepackage[letterpaper,top=2cm,bottom=2cm,left=3cm,right=3cm,marginparwidth=1.75cm]{geometry}

% Useful packages
\usepackage{amsmath}
\usepackage{graphicx}
\usepackage[colorlinks=true, allcolors=blue]{hyperref}

\title{Test 2: Investigación e Implementación del Framework XAF (Iteración del Test 1)}
\author{Paola Vilaseca}

\begin{document}
\maketitle

\section{Introducción al framwork XAF}
eXpressApp Framework (XAF) es un .NET appliction framework cread por DevExpress. XAF fue diseñada con el proposito de simplificar el desarrollo aplicaciones empresariales multi-plataforma. Con la integración de los controles de presentación de DevExpress, y con las bibliotecas ORM (Entity Framwork o XPO), el usuario puede construir aplicaciones de bases de datos tanto para WinForms como para ASP.NET (Blazor o Web Forms). 

Dentro de sus principales características se encuentran las siguientes \cite{componentsource2023}:

\begin{itemize}
    \item XAF permite al desarrollador crear aplicaciones que se adapten a las necesidades de la empresa ya que las mismas están diseñadas para funcionar tanto como aplicaciones de escritorio, como web o como móviles. 
    \item XAF se basa en el enfoque de desarrollo impulsado por modelos (Model-Driven Development), lo cual permite a los desarrolladores generar de manera automática interfaces de usuario (UI) y gestionar los datos de manera eficiente.  
    \item La arquitectura modular de XAF permite que se integren fácilmente más de 20 paquetes predefinidos y listos para usara. Dentro de estos paquetes se incluyen funcionalidades como análisis de datos, gráficos, mapeo, reportes, programación, seguridad, etc. Esto facilita la expansión y personalización de aplicaciones. 
    \item XAF facilita tanto las pruebas unitarias como las funcionales. Esto se debe gracias a su arquitectura modular, la cual permite realizar pruebas rápidas y ligeras. Adicional a esto, incluye un motor de pruebas multiplataforma para automatizar las pruebas en diferentes lenguajes, como C\#, y es compatible con sistemas de integración continua como Azure DevOps y NUnit.
    \item XAF permite que el desarrollador se enfoque en los datos, haciendo que este pase menos tiempo en el diseño de interfaz y pase más tiempo en la lógica empresarial. 
\end{itemize} 


\section{Cómo XAF puede mejorar el test 1}
XAF puede ser utilizado par mejorar o iterar sobre el test 1 ya que esta generaría las interfaces automaticamente, ahorrando tiempo en el diseño de pantallas. Por otro lado, XAF facilita la integración con base de datos relacionales y proporciona un sistema de seguridad robusto para gestionar usuarios y roles, optimizando la eficiencia y la seguridad del mismo. \cite{devexpress}  

A su vez, XAF podría facilitar la creación de aplicaciones multiplataforma para el test 1, simplificando el desarrollo móvil y web, ampliando su accesibilidad.  \cite{polyxer2024}

\section{Principales ventajas y desventajas de XAF}
Gracias a esta investigación se pudieron encontrar las siguientes ventajas de utilizar XAF \cite{programacionproXAF}:
    \begin{itemize}
        \item La generación de interfaces de usuario automáticas ahorra tiempo en el desarrollo.
        \item La utilización de patrones y estilos de codificación hacen que el código sea consistente. Esta estandarización facilita la colaboración entre desarrolladores y la compatibilidad con diferentes plataformas. Adicional a eso, el código consistente mejora tanto su legibilidad como su mantenimiento a largo plazo.  
        \item Ofrece un sistema integrado de gestión de usuarios y roles, aumentando la seguridad de las aplicaciones.
    \end{itemize}
    
Sin embargo, también se encuentran las siguientes desventajas  \cite{tithink2018}: 
    \begin{itemize}
        \item Si bien se ahorra tiempo de desarrollo, también se debe invertir tiempo de aprendizaje ya que antes de utilizar un framework, es necesario aprender su estructura y como se comunican sus componentes.
        \item Debido a sus constantes actualizaciones, se pueden crear problemas de compatibilidad o de seguridad.
        \item Los frameworks consumen mas recursos, haciéndolos ineficientes para proyectos de alto rendimiento.
        \item Si la aplicación es pequeña, una gran parte del código del framework no se utilizara.
    \end{itemize}
    
\section{Conclusiones}

En conclusión, el uso de XAF simplifica el desarrollo de aplicaciones al generar interfaces automáticamente y al ofrecer una arquitectura modular. Sin embargo, se debe invertir tiempo de aprendizaje y el uso del mismo requiere de más recursos.     


\bibliographystyle{alpha}
\bibliography{references}

\end{document}