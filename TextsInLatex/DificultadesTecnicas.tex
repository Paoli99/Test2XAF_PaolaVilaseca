\documentclass{article}


\usepackage[spanish]{babel}

% Set page size and margins
% Replace `letterpaper' with `a4paper' for UK/EU standard size
\usepackage[letterpaper,top=2cm,bottom=2cm,left=3cm,right=3cm,marginparwidth=1.75cm]{geometry}

% Useful packages
\usepackage{amsmath}
\usepackage{graphicx}
\usepackage[colorlinks=true, allcolors=blue]{hyperref}

\title{Test 2: Dificultades técnicas al integrar XAF}
\author{Paola Vilaseca}

\begin{document}
\maketitle

Durante la integración de XAF a la aplicación del test 1, se encontraron las siguientes dificultades: 

\begin{itemize}
    \item Debido a que es un framework con el que nunca trabaje, tuve que investigar cómo configurar correctamente la aplicación. Una parte que se me hizo un poco confusa al inicio era saber que características de configuración utilizar, por ejemplo saber si era mejor utilizar ORM o XPO junto con si usar Middle Tier Security era mejor que 2-Tier Security. Luego de investigar cómo estas funcionaban, procedí a analizar los pros y los contras de las opciones para ver que opción era mas conveniente usar para solucionar el problema.  
    \item Al inicio también tuve una pequeña confusión con respecto a la creación del sistema de autentificación. Intente crear mi propio sistema de autenticación desde cero, sin embargo cuando ejecuté la aplicación me di cuenta de que XAF ya proporcionaba de manera automática dicho sistema. Debido a esto, elimine el código que había creado para evitar posibles conflictos.
    \item Debido a que el XAF me proporciono automáticamente con un sistema de autentificación, tenia dudas sobre como es que se creaban los usuarios. Investigando un poco, pude ver que el framework también incluía usuarios por defecto, lo cual me termino facilitando la administración de roles y autenticación. 
    \item Inicialmente quería ver como funcionaba la aplicación antes de conectarla a una base de datos, lo cual me dificultó la sincronización ya que los datos no se actualizaban correctamente. Luego de intentar diferentes soluciones, opte por descartar el proyecto e iniciar de nuevo, asegurándome de conectar la base de datos al inicio para así evitar posibles conflictos en el futuro. 
\end{itemize}

En conclusión, el uso de XAF me puso algunos desafíos iniciales, especialmente a la hora de configurar e integrar funcionalidades claves como la autenticación y la conexión con la base de datos. Sin embargo, revisando la documentación observe cómo el framework es una herramienta que facilita el desarrollo de aplicaciones.

\end{document}